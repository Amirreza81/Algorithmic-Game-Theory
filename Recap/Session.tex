\documentclass{article}
\usepackage{enumitem}
\usepackage{floatrow}
\usepackage{multirow}
\usepackage{graphicx}
\usepackage{amsmath}
\usepackage{subcaption}
\usepackage{tikz}
\usepackage{mathtools}
%\def\hidesols{hide solutions} % uncomment this line to hide solutions
\usepackage[paperheight=20.1in,top=1in,bottom=1in,right=1in,left=1in,heightrounded]{geometry}
\usepackage{listings}
\usepackage{easylist}
\input{style}
%\input{../commons/style}

%\usepackage[dvipsnames]{xcolor}

\pagenumbering{gobble}
%	\definecolor{ivory}{rgb}{1.0, 1.0, 0.94}
%\pagecolor{ivory} 
\usepackage{watermark}
\usepackage{transparent}
\usepackage{setspace}

\begin{document}
\vspace{0.5cm}	

\سربرگ{یادآوری جلسه نهم}{یادآوری}{امیررضا آذری}
\vspace{0.5cm}
\setstretch{2}

\setlength{\abovedisplayskip}{0.5pt}
\setlength{\belowdisplayskip}{0.5pt}
\newtheorem{definition}{تعریف}
\newtheorem{theorem}{قضیه}
\newtheorem{example}{مثال}

در جلسه گذشته، به بررسی 
مسئله 
\textcolor{red}{
تقسیم منصفانه
}
یا همان 
\textcolor{red}{
برش کیک
}
پرداختیم. در این مسئله بر خلاف گذشته، رفتار استراتژیک بازیکن‌ها مورد بحث قرار نمی‌گیرد.
هدف این است که یک منبع مشترک بین تعدادی فرد با ترجیحات مختلف به صورت منصفانه تقسیم شود.
در ادامه سعی می‌کنیم منظورمان از واژه منصفانه را به صورت ریاضی مدل کنیم.

منبع‌هایی که قرار است بین افراد تقسیم شوند، می‌توانند به صورت آیتم‌های غیرقابل برش
باشند که نحوه تخصیص آن‌ها به این صورت است که یا کل یک بخش از منبع به یک فرد 
اختصاص داده ‌می‌شود یا هیچ قسمتی از آن اختصاص داده نمی‌شود.
در مقابل منابعی قرار دارند که به هر میزانی قابل قسمت هستند.
تمرکز ما در این درس بر اینگونه منابع است.
مسائلی که منبع آن‌ها به شکل پیوسته قابل تقسیم هستند به مسئله برش کیک معروف هستند.
که شامل موارد زیر هستند.
\begin{itemize}
    \item منبع یا کیک: آن را به صورت  \textcolor{red}{$\mathcal{C}$} نشان داده و به صورت بازه بین صفر و یک مدل می‌کنیم.
    \item مجموعه افراد: مجموعه افرادی که قرار است کیک میان آن‌ها تقسیم بشود و آن را با
    $N =\{a_1,\; a_2,\; \ldots,\; a_n\}$ نشان می‌دهیم.
    \item 
    
    تابع ارزش بازیکن $a_i$
    :
    آن را با $V_i$ نشان داده و به هر زیربازه $I$ از $[0, 1]$ یک ارزش نسبت می‌دهد.
    % \begin{center}
    %     $I:(0.1, 0.2), \;\;\; V_i\;(I) = 0.4$
    % \end{center}
    % که به این معناست که ارزش $I$ برای بازیکن \textcolor{red}{$i$}ام برابر $0.4$ است.
\end{itemize}
خواص فرض شده در مورد تابع $V_i$:
\begin{itemize}
    \item[1)]
    نرمال شدگی 
    \LTRfootnote{Normalization}:
    توابع ارزش طوری هستند که ارزش کل کیک برای هر نفر برابر \textcolor{red}{یک} است.
    \begin{center}
        $\forall_i,\; V_i\;(\mathcal{C}) = V_i\;([0, 1]) = 1$
    \end{center}

    \item[2)]
    تقسیم‌پذیری
    \LTRfootnote{Divisibility}:
     به ازای هر بازه $[x,\; y]$ و 
    $0 \leq \lambda \leq 1$،
    نقطه $z \in [x,\; y]$ وجود دارد که:
    \begin{center}
        $V_i\;([x,\; z]) = \lambda \; V_i\;([x,\; y])$
    \end{center}
    به طور مثال
    فرض کنید منبع پیوسته ما یک کیک است.
    یک تکه دلخواه از کیک را در نظر بگیرید.
    برای هر
    $
    \lambda \in [0, 1]
    $
    ،
    یک بخش از کیک وجود دارد که ارزش آن نسبت به تکه کیک اولیه برابر 
    $
    \lambda
    $
    است.
    \item[3)]
    جمع‌پذیری
    \LTRfootnote{Additivity}:
    توابع ارزش همگی جمع‌پذیر هستند.
    این به آن معنی است که اگر یک تکه از منبع را به دو قسمت تقسیم کنیم جمع ارزش هر کدام از بخش‌ها با ارزش تکه اولیه برابر خواهد بود.
    \[
    \begin{rcases*}
    V_i\;([0.2, \; 0.7]) &= 0/8 \\
    V_i\;([0.2, \; 0.4]) &= 0/3
    \end{rcases*} \longrightarrow V_i\;([0.4, \; 0.7]) = 0.5
    \]

    \item[4)]
    یکنواختی
    \LTRfootnote{monotonicity}:
    ارزش هیچ بخشی از منبع، منفی نیست.
\end{itemize}


تخصیصی 
متناسب 
\LTRfootnote{Proportionality}
است که هر فردی احساس کند ارزش سهم او حداقل 
{$\frac{1}{n}$} است.
تخصیص بدون رشک
\LTRfootnote{Envy free}
نیز حالتی است که هر فرد سهم خودش را به سهم دیگران ترجیح بدهد.
در واقع به ازای هر \textcolor{red}{$i$} و \textcolor{red}{$j$} داشته باشیم:
\begin{center}
    $V_i\;(A_i) \geq V_i\;(A_j)$
\end{center}
ما در تخصیص‌دهی فرض می‌کنیم
اولا تمام منبع بین افراد تقسیم می‌شود بدون آنکه قسمتی از آن تخصیص نیافته باقی بماند و همچنین  به هر بازیکن یک بازه تخصیص داده شود.

\begin{example}
    تقسیم کیک بین
    $2$
    نفر 
    (روش
    \lr{cut and choose}).
    در این روش ابتدا به نفر اول می‌گوییم که کیک را به دو قسمت با ارزش برابر تقسیم کند. سپس نفر دوم، قطعه‌ای را که بیشتر دوست دارد را انتخاب می‌کند. این تقسیم‌بندی، متناسب، بدون رشک، پیوسته و بدون هدر رفت می‌باشد.
\end{example}

\begin{example}
    تقسیم کیک بین 
    $
    3
    $
    نفر به صورت متناسب.
    ابتدا بازیکن اول، کیک را به
    $
    3
    $
    قسمت مساوی با ارزش $\frac{1}{3}$ برای خودش تقسیم می‌کند.
    برای بازیکن دوم و سوم یک قطعه، خوب محسوب می‌شود اگر حداقل به اندازه 
    $
    \frac{1}{3}
    $
    کل کیک برای آن بازیکن ارزش داشته باشد.
    اگر بین سه قطعه موجود، بازیکن‌های دوم و سوم
    هر کدامشان
    حداقل دو قطعه خوب داشته باشند، صبر می‌کنند
    تا طرف مقابل ابتدا قطعه مطلوب خود را انتخاب کند، بعد خودشان قطعه‌ای که در نظر دارند را انتخاب می‌کنند و بازیکن اول هم در آخر قطعه باقی مانده را انتخاب می‌کند.
    اگر هر دو بازیکن حداقل دو قطعه خوب داشتند مشابه حالتی که بازیکن دوم دو قطعه خوب داشت عمل می‌کنیم.
    اگر هر کدام از بازیکن‌های دوم و سوم 
    حداکثر یک قطعه خوب داشته باشند، آنگاه یک قطعه وجود دارد که بنظر برای هر دو بازیکن، قطعه غیرخوب محسوب می‌شود.
    آن قطعه را به بازیکن اول ‌می‌دهیم و بین بازیکن دوم و سوم مشابه حالت دو نفره مثال قبل عمل می‌کنیم.
\end{example}

\watermark{\centering\put(400,-1331){\includegraphics[scale=0.5]{pics/Bahar}}}
\end{document}